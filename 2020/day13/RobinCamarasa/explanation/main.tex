\documentclass[runningheads]{llncs}
\usepackage{graphicx}
\usepackage{cite}
\usepackage{amsmath}
\usepackage{amssymb}
\usepackage{verbatim}
\usepackage{hyperref}


\begin{document}
\title{
    Explanation Math of Part 2 of day 13 of the advent of code 2020
}

\author{
    Robin Camarasa
}

\maketitle

A full description of the problem is given at the following \textbf{\href{https://adventofcode.com/2020/day/13}{link}}. (To access part one you must finish part one first)

In a math way, the problem can be defined as follow:

\begin{align}
    \begin{split}
    t & \equiv 0 \mod{13}\\
    t + 3 & \equiv 0 \mod{41}\\
    t + 13 & \equiv 0 \mod{641}\\
    t + 25 & \equiv 0 \mod{19}\\
    t + 30 & \equiv 0 \mod{17}\\
    t + 42 & \equiv 0 \mod{29}\\
    t + 44 & \equiv 0 \mod{661}\\
    t + 50 & \equiv 0 \mod{37}\\
    t + 67 & \equiv 0 \mod{23}\\
    \end{split}
\end{align}

From these relations one can derive that, $t \equiv 0\mod{13} \iff \exists q \in \mathbb{Z} | t=13q$ and that 

\begin{align}
    \begin{split}
    13q & \equiv -30 \equiv -13 \mod{17}\\
    13q & \equiv -13 \mod{29}\\
    13q & \equiv -13 \mod{37}\\
    13q & \equiv -13 \mod{641}\\
    \end{split}
\end{align}

We can modify those relations as follow:

\begin{align}
    \begin{split}
        13 (q + 1) & \equiv 0 \mod{17}\\
        13 (q + 1) & \equiv 0 \mod{29}\\
        13 (q + 1) & \equiv 0 \mod{37}\\
        13 (q + 1) & \equiv 0 \mod{641}\\
    \end{split}
\end{align}

13, 17, 29, 37 and 641 being distinct primes implies that: $$\exists q' \in \mathbb{Z} | q + 1 = 17 \times 37 \times 29 \times 641 q'$$.

And $t = 13 \times 17 \times 37 \times 29 \times 641 q' - 13$. Looping over the values of $t$ gives the answer in less than 3 seconds on a laptop.


\end{document}
